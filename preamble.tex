\usepackage{cancel}
\usepackage{coffee4}
\usepackage[version=4]{mhchem}

\usepackage{fvextra} % needed for code wrapping
\DefineVerbatimEnvironment{Highlighting}{Verbatim}{breaklines,commandchars=\\\{\}}

\setCJKmonofont{Noto Sans Mono CJK SC}

\usepackage{listings}
\lstloadlanguages{TeX}
\definecolor{codegreen}{rgb}{0,0.6,0}
\definecolor{codegray}{rgb}{0.5,0.5,0.5}
\definecolor{codepurple}{rgb}{0.58,0,0.82}
\definecolor{backcolour}{rgb}{0.95,0.95,0.92}
\lstdefinestyle{mystyle}{
  backgroundcolor=\color{backcolour},   
  commentstyle=\color{codegreen},
  keywordstyle=\color{magenta},
  numberstyle=\tiny\color{codegray},
  stringstyle=\color{codepurple},
  basicstyle=\ttfamily\footnotesize,
  breakatwhitespace=false,         
  breaklines=true,                 
  captionpos=b,                    
  keepspaces=true,                 
  numbers=left,                    
  numbersep=5pt,                  
  showspaces=true,                
  showstringspaces=true,
  showtabs=true,                  
  tabsize=2
}
\lstset{style=mystyle}
\lstnewenvironment{TeXlstlisting}[2]{
	\lstset{
		caption=#1,
		label=#2,
		language=[LaTeX]TeX
	}
}{}

% Only displaying the list if there's an item available for that
\usepackage{xassoccnt}
\NewTotalDocumentCounter{totalfigures}
\NewTotalDocumentCounter{totaltables}
\NewTotalDocumentCounter{totallistings}
\DeclareAssociatedCounters{lstlisting}{totallistings}
\DeclareAssociatedCounters{figure}{totalfigures}
\DeclareAssociatedCounters{table}{totaltables}
\makeatletter
\renewcommand{\TotalValue}[1]{%
  \value{xassoccnt@total@#1}%
}
\let\@@latex@@listoffigures\listoffigures
\let\@@latex@@listoftables\listoftables
\let\@@latex@@lstlistoflistings\lstlistoflistings
\renewcommand{\listoffigures}{%
  \ifnum\TotalValue{totalfigures} > 0 
  \@@latex@@listoffigures%
  \fi
}
\renewcommand{\listoftables}{%
  \ifnum\TotalValue{totaltables} > 0 
  \@@latex@@listoftables%
  \fi
}
\renewcommand{\lstlistoflistings}{%
  \ifnum\TotalValue{totallistings} > 0 
  \@@latex@@lstlistoflistings%
  \fi
}
\makeatother

\usepackage{elements}
\usepackage{modiagram}
\usepackage{chemfig}
\usepackage{fdsymbol} % for nequal
\usepackage{circuitikz}

\pretolerance 16384

\usepackage{gensymb} % \celsius

\usepackage{siunitx}
\sisetup{
  input-decimal-markers = {.,},
  output-decimal-marker = {,},
  separate-uncertainty-units = repeat
}
% For "retain-explicit-decimal-marker"
\ExplSyntaxOn
\keys_define:nn { siunitx }
  { retain-explicit-decimal-marker .bool_set:N = \l__siunitx_number_explicit_decimal_bool }
\cs_gset_protected:Npn \__siunitx_number_parse_loop_main_end:NN #1#2
  {
    \tl_if_empty:NT \l__siunitx_number_partial_tl
      {
        \bool_if:NTF #2
          { \tl_set:Nn \l__siunitx_number_partial_tl { 0 } }
          {
            \bool_if:NT \l__siunitx_number_explicit_decimal_bool
              { \tl_set:Nn \l__siunitx_number_partial_tl { \empty } }
          }
      }
    \tl_put_right:Nx #1
      {
        { \exp_not:V \l__siunitx_number_partial_tl }
        \bool_if:NT #2 { { } }
        { }
      }
  }
\ExplSyntaxOff

\usepackage{titlesec}
\titleformat{\chapter}[display]
{\normalfont\LARGE\bfseries}{\chaptertitlename\ \thechapter}{18pt}{\LARGE}
\titleformat{\section}
{\normalfont\fontsize{16}{18}\bfseries}{\thesection}{1em}{}

\usepackage{pdflscape}

\setlength{\columnsep}{32pt}

\counterwithout{figure}{chapter}

\usepackage{multicol}
\setlength{\columnseprule}{1pt}
\newcommand{\multicolsbegin}{\begin{multicols}{2}}
\newcommand{\multicolsend}{\end{multicols}}
\usepackage{multicolrule}

% for fill-in-the-gaps-tests
\newlength{\blankwidth}
\newcommand{\blank}[1]{%
  \begingroup
  \setlength{\blankwidth}{0pt}%
  \ifmmode
    % mõõda praeguse matemaatikastiili järgi (display/text/script/scriptscript)
    \settowidth{\blankwidth}{$\mathchoice{\displaystyle #1#1#1}{\textstyle #1#1#1}{\scriptstyle #1#1#1}{\scriptscriptstyle #1#1#1}$}%
  \else
    \settowidth{\blankwidth}{#1#1#1}%
  \fi
  \makebox[\blankwidth]{\dotfill}%
  \endgroup
}

\usepackage{fontawesome5} % for hand pointers
\usepackage{wasysym} % for form elements

\usepackage{ragged2e}
\RaggedRight

\raggedbottom % for hindering to spread paragraphs over the page (@kottwitz_2011_latex(p. 95))
